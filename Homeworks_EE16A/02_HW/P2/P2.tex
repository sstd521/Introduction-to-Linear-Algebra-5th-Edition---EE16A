\documentclass{article}
\usepackage[left=3cm, right=3cm, top=3cm]{geometry}
\usepackage{amssymb}
\begin{document}

{\Large 2. (PRACTICE) Finding Charges from Potential Measurements}\\[.3cm]
$k\frac{Q_1}{r_{1,1}} + k\frac{Q_2}{r_{2,1}} + k\frac{Q_3}{r_{3,1}} = U_1 = k\frac{4+3\sqrt{5}+\sqrt{10}}{2\sqrt{5}}$ \\[.3cm]
$k\frac{Q_1}{r_{1,2}} + k\frac{Q_2}{r_{2,2}} + k\frac{Q_3}{r_{3,2}} = U_2 = k\frac{2+4\sqrt{2}}{\sqrt{2}}$ \\[.3cm]
$k\frac{Q_1}{r_{1,3}} + k\frac{Q2}{r_{2,3}} + k\frac{Q3}{r_{3,3}} = U_3 = k\frac{4+\sqrt{5}+3\sqrt{10}}{2\sqrt{5}}$ \\[.3cm]
Canceling $k$ and calculating out the distances would give us:\\[.3cm]
$\frac{Q_1}{\sqrt{2}} + \frac{Q_2}{\sqrt{5}} + \frac{Q_3}{2} = \frac{4+3\sqrt{5}+\sqrt{10}}{2\sqrt{5}}$ \\[.3cm]
$\frac{Q_1}{1} + \frac{Q_2}{\sqrt{2}} + \frac{Q_3}{1} = \frac{2+4\sqrt{2}}{\sqrt{2}}$ \\[.3cm]
$\frac{Q_1}{2} + \frac{Q2}{\sqrt{5}} + \frac{Q3}{\sqrt{2}} = \frac{4+\sqrt{5}+3\sqrt{10}}{2\sqrt{5}}$ \\[.3cm]
Using IPython to solve the linear equations,\\

$Q1 = -173.$\\[.15cm]

$Q2 = 490.$ \\[.15cm]

$Q3 = -163.$ \\[.15cm]




\end{document}