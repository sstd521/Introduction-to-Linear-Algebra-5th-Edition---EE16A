\documentclass{article}
\usepackage[left=3cm, right=3cm, top=3cm]{geometry}
\usepackage{amssymb}
\usepackage{amsmath}
\usepackage{mathrsfs}
\usepackage{xcolor}
\begin{document}

{\Large 3. Noise Cancelling Headphones} \\[.5cm]
{\color{red} (a) $\vec{m} + \mathbb{R}\vec{\gamma} + \vec{n}$} \\

The signal at the listener's ear is:
$$\vec{s} = \vec{m} + \mathbb{R}\vec{\gamma} + \vec{n}$$
{\color{red} (b) $\mathbb{R}\vec{\gamma} + \vec{n}$} \\

In order to have a signal at the ear that matches the original music signal perfectly, we should aim to minimize:
$$\mathbb{R}\vec{\gamma} + \vec{n}$$
{\color{red} (c) Implemented based on formula
$\vec{x} = (A^TA)^{-1}A^T\,\vec{b}$.} \\[.5cm]
{\color{red} (d)
$\vec{\gamma} =
	\begin{bmatrix}
         -0.0883 \\
 		-0.093 \\
 		-0.9184
 	\end{bmatrix}$ } \\

Using IPython, we have that:
$$\gamma_A = -0.0883$$
$$\gamma_B = -0.093$$
$$\gamma_C = -0.9184$$

which gives us that
\begin{center}
$\vec{\gamma} =
    \begin{bmatrix}
        \gamma_A \\
		\gamma_B \\
		\gamma_C
	\end{bmatrix} =
	\begin{bmatrix}
        -0.0883 \\
		-0.093 \\
		-0.9184
	\end{bmatrix}$
\end{center}
{\color{red} (e)} \\

For the three loaded sounds, music\_y is the song ``Hallelujah''; noise1\_y is full of static noise; noise2\_y is static noise plus some other noise (laughter and train whistle). \\

Then, adding the first noise to the signal literally imposes the static noise upon the song; adding the second noise does the same thing, with the extra bit of laughters and train whistles. \\

Only the origianl music (music\_y) is not full of static. Adding both noises would make the sound full of static, but the effect is especially obvious when adding the first noise.

\end{document}